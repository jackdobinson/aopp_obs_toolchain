
\subsection{Notation}
\begin{description}
	\item{Pupil Function (PF)} How the electric field is transmitted through a telescope to the exit pupil. Usually modelled as a real function of the amplitude of the E-field, but in principle can be complex and contain phase information.
	\item{Amplitude Transfer Function (ATF)} A scaled version of the PF such that frequencies with perfect transmission have a value of 1. Typically is the same as the PF.
	\item{Amplitude Spread Function (ASF)} \fft{\textrm{ATF}}, the optical system's response to \emph{coherent} light, i.e., phase differences do not matter.
	\item{Point Spread Function (PSF)} $|\textrm{ASF}|^2 = \fftconj{\textrm{ATF}}\fft{\textrm{ATF}} = \fft{\textrm{ATF} * \textrm{ATF}}$. The squared magnitude of the ASF, and the inverse fourier transform of the Optical Transfer Function (OTF). The spatial response of the system to \emph{incoherent} light.
	\item{Optical Transfer Function (OTF)} $\ifft{\textrm{PSF}}$ the frequency responce to \emph{incoherent} light. Is the autocorrelation of the ASF, and the fourier transform of the PSF.
	\item{Power Spectral Density (PSD)} The amount of power in each fequency component of a signal, equivalent to $\fft{\textrm{Autocorrelation of signal}}$

\end{description}



\subsection{Background}

Modelling MUSE's PSF is based on \cite{Fetick19}. Their approach is based upon modelling the power spectral density (PSD) of the atmosphere + adaptive optics (AO) system, finding the Optical Transfer Function (OTF) of that system, and combining it with the telescope's non-AO diffraction limited OTF to get the combined PSF.

Let $R(\x)$ be the PSF,
\begin{align}
	\hat{R}(\vec{k}) =& \fft{R(\x)}(\vec{k}) \quad \textrm{is the OTF of that system, and} \nonumber \\
	\hat{R}_T(\vec{k}) =& \fft{R_T(\x)}(\vec{k}) \quad \textrm{the OTF of the telescope} \nonumber \\
	\hat{R}_A(\vec{k}) =& \fft{R_A(\x)}(\vec{k}) \quad \textrm{the OTF of the atmosphere + AO system} \nonumber \\
	\therefore \hat{R}(\vec{k}) =& \hat{R}_T(\vec{k}) \hat{R}_A(\vec{k})
	\label{eq:full_otf}
\end{align}
where $\x$ is the light position at the exit pupil, and $\vec{k}$ is spatial wavenumber vector. Sometimes, $\x$ and $\vec{k}$ have other intepretations, but this is the simplest. Compare with \cite[eq 5, eq 9]{Fetick19}

The $\hat{R}_T$ is pretty easy to get, you just need to work out the pupil function (usually just a circle of 1's, or an annulus of 1's for a telescope with a secondary mirror), then autocorrelate it to get the OTF.

$\hat{R}_A$ is more difficult, we need to build some model for how the atmosphere and AO system influences the light. Fetick's method is to build up the PSD of this system. They use a Moffat function to model the AO-systems contribution, and kolmogorov turbulence to model the atmosphere's contribution, it's assumed that below some cutoff spatial frequency $f_{AO}$, the adaptive optics are the dominant factor, and above that frequency the AO-system can't correct on such small scales so the atmosphere is the dominant factor. Therefore their model of the PSD is \cite[see][eq 11, eq 2, eq 3, eq 10]{Fetick19}

\begin{align}
	W_\phi(\vec{f}) = \left[ 
		\frac{\beta -1}{\pi \alpha_x \alpha_y}
		\frac{M_A(f_x, f_y)}{ 1 - \left( 1 + \frac{f_{AO}^2}{\alpha_x \alpha_y}\right)^{(1-\beta)}} + C
		\right]_{f \leq f_{AO}}
		+
		\left[W_{\phi,\textrm{kolmo}}(\vec{f}) \right]_{f>f_{AO}}
		\label{eq:fetick11}
\end{align}
where
\begin{align}
	M(x,y) =& \frac{A}{\left(1 + \left(\frac{x}{\alpha_x}\right)^2 + \left(\frac{y}{\alpha_y}\right)^2 \right)^\beta} \label{eq:moffat}\\
	A =& \frac{\beta -1}{\pi \alpha_x \alpha_y} \nonumber
\end{align}
and
\begin{align}
	W_{\phi,\textrm{kolmo}}(\vec{f}) = 0.023 r_0^{-5/3} f^{-11/3}
	\label{eq:kolmo_psd}
\end{align}
with $\alpha_x$, $\alpha_y$, $\beta$ being parameters to the Moffat function \cite{Moffat69}. $C$ is a constant offset usually set so there is approximate continuity between the two different parts of the equation. $r_0$ is the \emph{Fried Parameter}, and $\vec{f}$ is spatial frequency, with $f$ being its magnitude. 

The power of $f$ in \eqref{eq:kolmo_psd} ($-11/3$) and the coefficient ($0.23$) are due to the dimensionality of the turbulence, with a power of $-11/3$ for 3D, $-8/3$ for 2D, and $-5/3$ for 1D. The coefficient is fairly arbitrary, as it's the relative sizes of the AO and atmospheric parts that is important, but a value of $0.000023 \times 10^D$, where $D$ is the dimensionality of the turbulence, gives a good continuity between the two parts of equation \eqref{eq:fetick11}. I.e.,
\begin{align}
	W_{\phi,\textrm{kolmo}}(\vec{f}; D) = 2.3\times10^{D-5} r_0^{\frac{-5}{3}} f^{\frac{3D-2}{3}}
	\label{eq:kolmo_psd_dim}
\end{align}

Equation \eqref{eq:kolmo_psd_dim} was found by working through the equations for Kolmogorov turbulence for other dimensions, and comparing the $f>f_{AO}$ parts of modelled PSFs with example PSFs.

Finally, getting $\hat{R}_A$ from \eqref{eq:fetick11} can go two ways. In \cite{Fetick19} they use the relationship
\[
	\ifft{W_\phi(\vec{f})} = B_\phi(\rho)
\]
and
\begin{align}
	\hat{R}_A = e^{-B_\phi(0)} e^{B_\phi(\rho)}.
	\label{eq:psf_a}
\end{align}
However, it's also mentioned \cite[section 2.3]{Fetick19} that for extreme AO correction, the PSF = PSD. I assume this is because you can expand the exponential as a series. Therefore to first order, we can just use
\begin{align}
	\hat{R}_A = \ifft{W_\phi(\vec{f})}.
	\label{eq:psf_a_approx}
\end{align}
I've programmed in both versions to the modelling code, but I've had most success wth \eqref{eq:psf_a_approx} as it seems to fix the problem of the full treatment having a large delta-function-like spike that is not found in example PSF observations.

Once you have the value of $\hat{R}_A$, use \eqref{eq:full_otf} to get the full OTF and from that find the PSF.

One more thing to consider is the impact of supersampling on the modelled PSF. \cite{Fetick19} mentions this briefly, but I found bad fits to example PSF observations when running the calculations at MUSE's native resolution. The modelled PSFs look much better when running the caclulations by supersampling at factors of 2x or even 4x and smoothing the results to the native resolution.


\subsection{Why I Think Fetick's Treatment Is Being Weird}
So, when I've used \eqref{eq:psf_a} to calculate $\hat{R}_A$ I've found a curious problem. The central lobe of the PSF is dominated by a very large central spike, this is not found in any of the observational PSFs and I couldn't find a good way to remove it without arbitrarily altering \eqref{eq:psf_a}. Supersampling starts to mitigate the problem, and practically fixes it when you use enough of it.

\begin{aside}
	The supersampling I am refering to here is in spatial frequency. I.e., when calculating the Pupil Function use many more pixels, but the size of each pixel is the same. That way, when the PF is fourier transformed you have better resolution in spatial frequency as the maximum frequency is the same (its the spatial frequency of a single pixel), but there are many more data samples. That way you get better spatial frequency resolution.
\end{aside}

But why does this problem occur in the first place? I think it's due to how the PSD is parameterised in \eqref{eq:fetick11}. From reading books and articles about Kolmogorov turbulence and its interaction with optics (see \cite{AO_online_book, Durbin21, McMurty_OnlineTurbulenceCourse, Andrews19, Tyson_2012, Coulman85, born_wolf_2019, Lahiri16, Schroeder00, Keating02, roddier_1999}), what we are measuring with the PSF is the intensity of light that passes through our optical system. So, in 1D for simplicity,
\begin{align*}
	I(x) =& \left< |\psi(x,t)|^2 \right>_\textrm{av} \\
		=& \left< \psi^\star(x,t)\psi(x,t) \right>_\textrm{av} \\
		=& \left< |\psi_0|^2 e^{ik(ct-x)} e^{ik(-ct+x)}\right>_\textrm{av} \\
		=& \psi_0^2 \quad \textrm{assuming $\psi_0$ is real}
\end{align*}
where $I(x)$ is the intensity of light, $\psi(x,t)$ is the wavefunction of the light's electric field for a light wave with wavenumber $k$, and $\left< ... \right>_\textrm{av}$ means to take the average of the quantity inside the brackets over whatever space is applicable (e.g. time average). For combination of light waves, you get the following relationship (simplifying the notation to avoid having to write the brackets all the time):
\begin{align}
	I_k(x) =& \left<  |\psi(x,t)|^2 \right>_{k} \quad \textrm{avg. over wavenumber} \nonumber \\
	I_{k}(x) =& I_0(x) + I_1(x) + ... + I_n(x) \quad \textrm{For incoherent light,} 
	\label{eq:intensity_incoherent}\\
	I_{k}(x) =&  |\psi_0^(x,t) + \psi_1^(x,t) + ... + \psi_n^(x,t)|^2  \quad \textrm{For coherent light.} \nonumber \\
		=&  (\psi_{0,0} e^{ik_0(ct-x)} + ... + \psi_{0,n} e^{ik_n(ct-x)})(\psi_{0,0} e^{ik_0(-ct+x)} + ... + \psi_{0,n} e^{ik_n(-ct+x)})  \nonumber \\
		=& \sum_i \sum_j \psi_{0,i}\psi_{0,j}e^{i k_i x'}e^{i k_j (-x')} \nonumber \\
		=& \sum_i \psi_{0,i} e^{i k_i x'} \sum_j \psi{0,j}e^{i k_j (-x')} \nonumber \\
		=& \fft{\psi(x')}(k) \fft{\psi(-x')}(k) \nonumber \\
		=& \fft{\psi(x')}(k) \fftconj{\psi(x')}(k) \quad \textrm{which is the autocorrelation} \nonumber \\
		=& \sum_i \sum_j \psi_{0,i}\psi_{0,j}e^{i \,{\delta k}_{ij} \, x'} 
	\label{eq:intensity_coherent}
\end{align}
where $x'=ct-x$, and ${\delta k}_{ij}=k_i - k_j$. For incoherent lignt, the phase differences in \eqref{eq:intensity_coheret} cancel out (they are uncorrelated, so averaging. For coherent light, the phase differences are still important.

In terms of atmospheric turbulence affecting optics, the intensity is written in terms of the \emph{structure function} of the phase difference. A structure function describes how the differences between two quantities varies over their domain, e.g.
\[
	D_{gf}(t) = \int_{-\infty}^{\infty} |g(t'+t) - f(t')|^2 dt'
\]
characterises the difference between two functions $f(t)$ and $g(t)$ over their domain. Also, when $g=f$ and $t=0$, we have $D_{gg}(0) = 0$. This case is abbrivated to $D_g(t)$.

Turbulence affects the distribution of practically all of the properties of the atmosphere, one of these is the refractive index, which changes the optical path length, which leads to differences in the phase of light from a source when observed at the ground. As noted in \eqref{eq:intensity_coherent} earlier, the intensity of light is just its autocorrelation. When considering the effects of turbulence on the light from a point source, we just care about how correlated the light is over our detector. Let $D_\phi(r)$ be the structure function that describes how phase of light, $\phi$, varies with separation from some point described by $r=0$. Therefore,
\begin{align}
	B_\psi(r) =& \left< \psi(x)\psi^\star(x+r) \right>_\phi 
		 \nonumber \\
		=& e^{-\frac{1}{2}D_\phi(r)} \\
		=& e^{B_\phi(r) - B_\phi(0)}
		\label{eq:light_phase_correlation}
\end{align}
describes how the intensity of light changes over different length-scales $r$. I.e. $B_\psi(r)$ is the fourier transform of the PSF. I.e., The PSF describes how the intensity of light varies in space, $B_\psi(r)$ descibes how the intensity of light varies as the spatial scale increases. Linking to \eqref{eq:psf_a}, $B_\psi(r) = \hat{R}_A(\vec{k})$, but just the Kolmogorov turbulence part.

Ok, as $D_\phi(r)$ describes the difference in phase from some point $r=0$, we can safely assume that $D_\phi(0)=0$ by definition, and that in most cases $D_\phi(r) \rarrow \infty$ as $r \rarrow \infty$. Therefore, from \eqref{eq:light_phase_correlation} we can see that $B_\psi(0)=1$ and $B_\psi(r) \rarrow 0$ as $r \rarrow \infty$. So there should not be a delta-function-like spike at the center of the PSF (as that requires a constant value as $r \rarrow \infty$).

So, why the difference? Well, if we expand the exponential into its series representation
\begin{align*}
	B_\phi(r) =& e^{-B_\phi(0)} \sum_{n=0}^{\infty} \frac{B_\phi^n(r)}{n!} \\
		=& e^{-B_\phi(0)} + e^{-B_\phi(0)} B_\phi(r) + e^{-B_\phi(0)} frac{B^2_\phi(r)}{2} + ...
\end{align*}
we can see that as $B_\psi(r) = \hat{R}_A(\vec{k})$, and $B_\phi(r) = \ifft{W_\phi(f)}$,
\begin{align*}
	R_A(\x) =& e^{-B_\phi(0)} \left(\ifft{1} + \ifft{B_\phi(r)} + \textrm{smaller terms} \right), \\
		=& e^{-B_\phi(0)}(\delta(\x) + \ifft{\ifft{W_\phi(f)}}(\x) + \textrm{smaller terms} \\
		=& e^{-B_\phi(0)}(\delta(\x)) + W_\phi(\x)) + \textrm{smaller terms}
\end{align*}
where $e^{-B_\phi(0)}$ is a constant, and due to the double inverse fourier transforms $\x = -\vec{f}$.

As $\delta(\x)$ is an infinitestimal quantity, as we increase our numerical precison by super-sampling it will become less and less important. Similarly, the other terms of the exponential series will also be less important, and even if they never tend to zero as a first approximation we can just use \eqref{eq:psf_a_approx} which is much more computationally efficient anyway.





